
\newdualentry{AandG}
{AandG}
{Acquisition and Guiding}
{Process of observation. After slewing the telescope to a given field, the field center is acquired using one (or some) of star(s) in the field.}


\newdualentry{ADC}
{ADC}
{Atmospheric Dispersion Corrector}
{A part of WFC comprised of two glasses. By offsetting them in Y (gravitational) direction, atmospheric dispersion is corrected.}


\newdualentry{AG}
{AG}
{Auto Guide}
{Process of observation. While observing, the target field should be tracked automatically. By measuring centroid of star(s) on designated camera(s), tracking errors are calculated.}


\newdualentry{AGC}
{AGC}
{Acquisition and Guidance Camera}
{6 cameras are accommodated in the field-of-view edge for A\&G and AG. An additional glass is accommodated in front of half area of camera sensor to have two focal position. ASIAA has a responsibility for their development.}


\newdualentry{agcc}
{agcc}
{Acquisition and Guidance Camera Controller}
{Software module to control AGCs and measure centroid and FWHM of the spots. ASIAA has a responsibility for its development.}


\newdualentry{AIT}
{AIT}
{Assembly and Integration Tests}
{After shipment from each laboratory, an instrument should be assembled, integrated (again) and tested in its commissioning process. Subaru call this process AIT.}


\newdualentry{APC}
{APC}
{Auxiliary Power/control Electronics}
{This box is accommodated on the side of PFI, and supplies power and network for PFI components. The box also provides telemetry. ASIAA has a responsibility for its development.}


\newdualentry{ASIAA}
{ASIAA}
{Academia Sinica Institute of Astronomy and Astrophysics, Taiwan}
{PFS collaborator. ASIAA has responsibility for development of PFI and MCS.}


\newdualentry{BCB}
{BCB}
{Bulkhead Connector Box}
{The PFI connector box accommodated to the cable wrapper (CWA). The power, network and coolant from Subaru are induced at this box. ASIAA has a responsibility for its development.}


\newdualentry{bcu}
{bcu}
{Blue Camera Unit}
{Software module to operate BCU, in particular vacuum, cooling down, and telemetry.}


\newdualentry{CAL}
{CAL}
{Calibration Lamp}
{On the top of PFI, calibration lamps are accommodated to illuminate homogeneously the screen on the ceiling of the dome. Two arc lamps and one continuum lamp are attached. PU has a responsibility for its development.}


\newdualentry{Caltech}
{Caltech}
{California Institute of Technology, USA}
{PFS collaborator. It is also called CIT. Caltech has a responsibility for developing fiber positioners 'Cobra'.}


\newdualentry{CB2F}
{CB2F}
{Control Building 2nd Floor}
{The place at which main control infrastructure will be accommodated.}


\newdualentry{CIT}
{CIT}
{California Institute of Technology, USA}
{PFS collaborator. It is also called Caltech. CIT has a responsibility for developing fiber positioners 'Cobra'.}


\newdualentry{CLA}
{CLA}
{Cell Assemblies of Tower Connector of Cable C}
{Assembly of cable C. CLA consists of two MTP ferrules, and 11 cells are accommodated in one Tower Connector unit. LNA has a responsibility for development.}


\newdualentry{COB}
{COB}
{Cobra Optical Bench}
{PFI component. Cobra modules for science fibers, fixed fiducial fibers and AG cameras are mounted on this bench. CIT has a responsibility for its development.}


\newdualentry{COBRA}
{COBRA}
{COBRA positioners}
{Also written as 'Cobra'. Two-staged piezo motor by New Scale for fiber configuration. CIT has a responsibility for purchase. JPL has a responsibility for developing control electricity and software.}


\newdualentry{CPPC}
{CPPC}
{Chinese PFS Participant Consortium}
{PFS collaborator. As of May 2017, 6 institutes (Shanghai Jiao Tong University, National Astronomical Observatories of China, Tsinghua University, The University of Science and Technology of China, Xiamen University, and Peking University) join to this consortium.}


\newdualentry{CWA}
{CWA}
{Cable Wrapper Assembly}
{Rotating mechanism on the top pf PFI in order to route the power, network and coolant lines. ASIAA has a responsibility for its development.}


\newdualentry{DPS}
{DPS}
{DC Power Supply}
{The device in Auxiliary Power/Control Electronic (APC) box of PFI to convert AC power to DC power for other PFI devices.}


\newdualentry{DRP}
{DRP}
{Data Reduction Pipeline}
{Software for data processing. There are two kinds of pipelines according to reduction stage. 2D-DRP, developed by PU, extracts spectra calibrated for both wavelength and flux. 1D-DRP by LAM measures spectral parameters such as redshift.}


\newdualentry{emp}
{emp}
{Environment Monitoring module using rasPi}
{Software to control telemetry system for Spectrograph Clean Room (SCR).}


\newdualentry{enu}
{enu}
{Entrance Unit}
{Software to control Spectrograph components other than Camera Units, such as BIA and slit. LAM has a responsibility for its development.}


\newdualentry{ETC}
{ETC}
{Electronics Thermal Control}
{PFI assembly consisting of heat exchangers and coolant lines in order to limit heat dissipation into the telescope dome air.}


\newdualentry{ETH}
{ETH}
{Ethernet Hub}
{Ethernet Hub switch accommodated in Auxiliary Power/Control Electronics (APC) box.}


\newdualentry{ETS}
{ETS}
{Exposure Targeting Software}
{Software to allocate fibers to a provided target list. MPA has a responsibility for its development.}


\newdualentry{FBS}
{FBS}
{Fiducial Fiber Beam Splitter}
{Optics element for fiducial fiber illuminator. FBS is used to reflect LED light into fiducial fibers, and to transmit light from fiducial fibers to their viewing camera.}


\newdualentry{fcc}
{fcc}
{Field Center Camera}
{Software to control field center camera of PFI. ASIAA has a responsibility for its development.}


\newdualentry{FEL}
{FEL}
{Field Element}
{PFI component. In order to have the similar optical path to HSC, a glass plate of 54 mm thick is accommodated at the bottom of PFI.}


\newdualentry{FEM}
{FEM}
{Field Element Mounts}
{PFI component to mount the field element (FEL) to the Cobra bench (COB). CIT has a responsibility for its development.}


\newdualentry{FFC}
{FFC}
{Fixed Fiducial Camera}
{PFI component to take image of targets acquired on fixed fiducial fibers. ASIAA has a responsibility for its development.}


\newdualentry{FFF}
{FFF}
{Fixed Fiducial Fibers}
{Fibers fixed on the Cobra bench (COB) used as a reference of fiber position. There are 96 FFF in total, 84 for science fibers, and 12 for AG cameras.}


\newdualentry{FFI}
{FFI}
{Fiducial Fiber Illuminator}
{PFI component to back-illuminate fixed fiducial fibers and take image of the light coming through the fiducial fibers.}


\newdualentry{FIS}
{FIS}
{Fiducial Fiber Illumination Source}
{The system to back-illuminate fixed fiducial fibers.}


\newdualentry{fms}
{fms}
{Fiber connection Monitoring System}
{FOCCoS software to illuminate monitoring fibers, to take their image, and judge the connection among Cable A, B and C.}


\newdualentry{FOCCoS}
{FOCCoS}
{Fiber Optical Cable and Connector System}
{PFS instrument subsystem which delivers light from PFI at the prime focus to SpS  on the IR4 floor. LNA has a responsibility for FOCCoS development.}


\newdualentry{FPRD}
{FPRD}
{Functional Performance Requirements Document}
{One of the documents at Preliminary Design Review. FPRD describes all requirements for PFS instrument to satisfy scientific requirements. For system-level requirements, the latest version is available on PBworks.}


\newdualentry{FPS}
{FPS}
{Fiber Positioning Sequencer}
{PFI software to operate fiber configuration. FPS controls MCS exposure, derives fibers positions from measured centroids, and sends them with command to MPS to move fibers. ASIAA has a responsibility for its development.}


\newdualentry{FUA}
{FUA}
{Fixed Upper Assembly}
{One of the two main PFI structure. FUA is fixed to POpt2 and accommodates cable wrapper assembly,(CWA), modular plate for fiber cables (MP3), and calibration lamp and so on.}


\newdualentry{fvc}
{fvc}
{Fiber Viewing Camera}
{PFI software to take images of fixed fiducial fibers. fvc also turn on/off LED to illuminate fixed fiducial fibers. ASIAA has a responsibility for its development.}


\newdualentry{Gen2}
{Gen2}
{Subaru Observation Support System Generation 2 (SOSS Gen2)}
{The Subaru observation control system (OCS). It launched in 2011.}


\newdualentry{HSC}
{HSC}
{Hyper Suprime-Cam}
{Wide field (~1.5 degree in diameter) camera with more than 100 CCDs at Subaru prime focus. HSC SSP, which started in 2014, is sister survey of PFS in the SuMIRe project.}

\glsunset{HSC}

\newdualentry{ICD}
{ICD}
{Interface Control Document}
{If there is an interface between different developer institutes, they discuss it using ICD.}


\newdualentry{ICS}
{ICS}
{Instrument Control Software}
{Software packages to operate PFS instrument. Individual actuators, which control smaller components, exchange command, status with each other via messaging hub systems (MHS).}


\newdualentry{IIC}
{IIC}
{Instrument Interface and Controller}
{ICS software module which organizes operation of PFS subsystems in an exposure sequence, in harmony with the telescope.}


\newdualentry{IPMU}
{IPMU}
{Kavli Institute for the Physics and Mathematics of the Universe, the University of Tokyo}
{PFS collaborator (PI) taking initiative to proceed the project. IPMU also has a responsibility to develop software interfacing between PFS and Subaru, and execute system commissioning.}


\newdualentry{IR3}
{IR3}
{IR-side 3rd floor in the dome building }
{The place on which the UPS, chillers for spectrograph modules and the spectrograph clean room (SCR) will be accommodated.}


\newdualentry{IR4}
{IR4}
{IR-side 4th floor in the dome building }
{The place on which the spectrograph clean room (SCR) will be constructed. The floor used to be called TUE-IR.}


\newdualentry{JHU}
{JHU}
{Johns Hopkins University, USA}
{PFS collaborator. JHU has a responsibility for development of Spectrographs System.}


\newdualentry{JPL}
{JPL}
{Jet Propulsion Laboratory, USA}
{PFS collaborator. JPL has a responsibility for control system of Cobra positioners.}


\newdualentry{LAM}
{LAM}
{Laboratoire d'Astrophysique de Marseille, France}
{PFS collaborator. LAM leads development of Spectrographs Systems (SpS), and has a responsibility for development of 1D-data reduction pipeline (DRP).}


\newdualentry{LNA}
{LNA}
{Laborato\'rio Nacional de Astrofi\'sica, Brazil}
{PFS collaborator. LNA has a responsibility for development of Fiber Optical Cable and Connector System (FOCCoS).}


\newdualentry{MAC}
{MAC}
{MLP1 AG Converter}
{Software as an interface between PFS Auto Guide Cameras (AGCs) and Subaru Mid-Level Processors 1 (MLP1). MAC calculates tracking error using information from AGCs and send them to MLP1. MAC also sends AGCs image to guide viewer after converting proper format. IPMU has a responsibility for its development.}

\glsunset{MAC}

\newdualentry{MCPS}
{MCPS}
{Master Controller Power Supply}
{Device of Positioner Remote Electronics (PRE). MCPS supplies DC power to controller modules of Cobra positioners.}


\newdualentry{MCS}
{MCS}
{Metrology Camera System}
{PFS instrument subsystem installed to the Cassegrain focus. MCS takes image of back-illuminated fibers, and measures their centroids. ASIAA has a responsibility for its development.}


\newdualentry{meb}
{meb}
{MCS Electronic Box interface}
{Software to control power supply and telemetry system for MCS. ASIAA has a responsibility for its development.}


\newdualentry{MHS}
{MHS}
{Messaging Hub System}
{Server for various software modules (actors) to exchange their 'messages' such as statuses, commands. MHS provides a client library for network communication, and analysis of messages. PFS uses 'tron' from SDSS.}


\newdualentry{MLP1}
{MLP1}
{Mid-Level Processors 1}
{Subaru computer system which controls the entire telescope drivers and sensors. Tracking error in azimuth and elevation axes, which is measured by PFS software MAC, is sent to MLP1.}

\glsunset{MLP1}

\newdualentry{MP1}
{MP1}
{Modular Plate 1}
{One of the three plate to route fiber cables inside of PFI, from tower connector at the top to focal plane at the bottom. MP1 is the bottom-most one.}


\newdualentry{MP2}
{MP2}
{Modular Plate 2}
{One of the three plate to route fiber cables inside of PFI, from tower connector at the top to focal plane at the bottom. MP2 is the middle one.}


\newdualentry{MP3}
{MP3}
{Modular Plate 3}
{One of the three plate to route fiber cables inside of PFI, from tower connector at the top to focal plane at the bottom. MP3 is the top-most one, and fixed to the cable wrapper.}


\newdualentry{MPA}
{MPA}
{Max-Planck-Institut fuer Astrophysik, Garching, Germany}
{PFS collaborator. MPA has a responsibility for developing software package Exposure Targeting Software (ETS).}


\newdualentry{MPS}
{MPS}
{Movement Planning Software}
{Software to control Cobra positioners. MPS is operated by Fiber Positioning Sequencer (FPS). JPL has a responsibility for its development.}


\newdualentry{NAOJ}
{NAOJ}
{National Observatory of Japan}
{PFS collaborator. Parent organization of Hawaii Observatory (Subaru).}


\newdualentry{ncu}
{ncu}
{NIR Camera Unit}
{Software module to operate NCU, in particular vacuum, cooling down, and telemetry. JHU and PU have a responsibility for its development.}


\newdualentry{OBCP}
{OBCP}
{Observation Control Processor}
{Computers to control Subaru instruments. In PFS case, ICS works as OBCP.}


\newdualentry{OCS}
{OCS}
{Observation Control System}
{OCS is used to operate observations, interfacing of users (operators and/or observers) and various computers controlling telescope and instruments. the command by a user transferred to other control systems. Current OCS is called 'Gen2'.}


\newdualentry{OWS}
{OWS}
{Observation Workstation}
{Computers at Subaru telescope (and remote observation room) to send commands to telescope and instruments via Gen2.}


\newdualentry{PBA}
{PBA}
{Positioner Bench Assembly}
{PFI assembly for mounting Cobra positioner and AG cameras. PBA also consists of field element and AG cameras and their mount structure.}


\newdualentry{PDE}
{PDE}
{Positioner Driver Electronics}
{Electronics board that drives one row of positioners (28 or 29) in a module. PDEs are assembled to Cobra module. JPL has a responsibility for its development.}


\newdualentry{peb}
{peb}
{PFI Electronic Box interface}
{Software to control power supply and telemetry system for PFI. ASIAA has a responsibility for its development.}


\newdualentry{PFI}
{PFI}
{Prime Focus Instrument}
{PFS instrument installed to the Subaru Prime Focus positioning fibers on astronomical objects. PFI has not only fibers and their positioners on the focal plane, but also a variety of cameras for field acquisition and guiding, viewing fiducial fibers and small region at field center. ASIAA, CIT, JPL and LNA have a responsibility for its development.}


\newdualentry{PFS}
{PFS}
{Prime Focus Spectrograph}
{This project.}


\newdualentry{PFU}
{PFU}
{Prime Focus Unit}
{General name for Subaru-mounting modules for the instrument at prime focus. For PFS, POpt2 is the case.}


\newdualentry{PMA}
{PMA}
{Positioner Module Assembly}
{An assembly of fiber (Cable C), microlens, fiber positioner and driver boards. One module is designed to accommodate 57 fibers. PMA is also called 'Cobra module'. JPL, CIT and LNA has a responsibility for is development.}


\newdualentry{PMC}
{PMC}
{Positioner Master Controller}
{Device to control driver boards of cobra positioners.}


\newdualentry{PO}
{PO}
{(PFS) Project Office}
{Group made-up from members at IPMU and NAOJ to manage the PFS project.}


\newdualentry{POF}
{POF}
{Positioner Frame}
{Interface plate with instrument rotator of POpt2. POF also has a structure to mount cobra optical bench (COB). ASIAA has responsibility for its development.}


\newdualentry{POpt2}
{POpt2}
{Prime focus unit for Optical observation \#2}
{Housing structure for Prime Focus Instrument. It consists of hexapod which positions wide field corrector (WFC), and instrument rotator.}

\glsunset{POpt2}

\newdualentry{PPSx}
{PPSx}
{Positioner Power Supply x}
{x is 1, 2, or 3. Power Supply for Cobra positioners. PPS1 and 2 are accommodated to Positioner Remote Electronics Enclosure A, while PPS3 to PRE Enclosure B.}


\newdualentry{PRA}
{PRA}
{Parking Room Assembly}
{Parking space on the top pf PFI for Fiber cables when the cables are not  connected to cable B at the prime focus. ASIAA and LNA have a responsibility for its development.}


\newdualentry{PRE}
{PRE}
{Positioner Remote Electronics}
{PFI electronics to operate Cobra positioners remotely. PRE has two enclosures to accommodate power supplies and controller module. JPL and CIT have a responsibility for its development.}


\newdualentry{PRE-A}
{PRE-A}
{PRE Enclosure A}
{An enclosure for positioner remote electronics, accommodated on the side wall of PFI. Two power supplies for Cobra positioners (PPS1 and PPS2) are enclosed.}


\newdualentry{PRE-B}
{PRE-B}
{PRE Enclosure B}
{An enclosure for positioner remote electronics, accommodated on the side wall of PFI. One power supply for Cobra positioners (PPS3) and for positioner controller (MCPS), and positioner controller are enclosed.}


\newdualentry{PSC}
{PSC}
{Power Switch Controller}
{Device to turn on and off powers of AG cameras and other PFI components. PSC is accommodated in the Auxiliary Power/control Electronics box (APC).}


\newdualentry{PSci}
{PSci}
{(PFS) Project Scientist}
{Scientists who join PFS SSP survey. There are three main working groups: cosmology, galaxy evolution and AGN, and galaxy archaeology.}


\newdualentry{PSE}
{PSE}
{(PFS) Project System Engineer}
{System Engineer for developing PFS instrument. Each developer institute has one or two PSE(s). They have bi-weekly teleconf. to discuss progress, issues and plans of development,}


\newdualentry{PU}
{PU}
{Princeton University, USA}
{PFS collaborator. PU is a one of the SpS and PFI development team with hardware aspect. For software development, PU has a responsibility for 2D-DRP and database.}


\newdualentry{rcu}
{rcu}
{Red Camera Unit}
{Software module to operate RCU, in particular vacuum, cooling down, and telemetry. JHU and PU have a responsibility for its development.}


\newdualentry{RLA}
{RLA}
{Rotating Lower Components}
{One of the two main PFI structure mounted to instrument rotator. The main components of RLA are fibers and their positioners, AG cameras and field elements.}


\newdualentry{SAS}
{SAS}
{Status Archive System}
{Software which collects all PFS instrument status, and sends some of them to Subaru Telemetry System. SAS will be designed to analyze status for error handling etc. IPMU has a responsibility for its development.}


\newdualentry{SCP}
{SCP}
{Shaft Coupler}
{Part of positioner module assembly to connect the shaft of Cobra positioner to fiber arm. SCP also provides hard stop on the second stage motion of Cobra.}


\newdualentry{SCR}
{SCR}
{Spectrograph Clean Room}
{Dedicated clean room on the IR4 floor for Spectrograph Modules. The temperature is controlled in the room.}


\newdualentry{SOCD}
{SOCD}
{Science and Operational Concept Document}
{One of the documents at Preliminary Design Review. SOCD describes scientific motivations for developing PFS instrument and its operation modes.}


\newdualentry{SOSS}
{SOSS}
{Subaru Operations Software System}
{First-generation software system for Subaru operation. SOSS provided user interface, scheduler, data archiving and so on. SOSS was replaced to Gen2 in 2011.}


\newdualentry{SpS}
{SpS}
{Spectrograph System}
{PFS instrument subsystem comprised of 4 spectrograph modules. SpS is designed to be accommodated in the clean room on IR4 floor. LAM. PU, JHU, LNA and IPMU are responsible for development.}


\newdualentry{SPT}
{SPT}
{Survey Planning \& Tracking software}
{Software package for operating PFS SSP survey. Detailed design and functionality are under discussion by task-force, as of early 2017.}


\newdualentry{ssc}
{ssc}
{SNMP status collector}
{ICS actor which collects statuses of PFS control infrastructures via SNMP and push the statuses through MHS.}


\newdualentry{SSP}
{SSP}
{Subaru Strategic Program}
{SSP is extremely large survey (~more than 100 nights for a several years) which take advantages of unique instrument and Subaru telescope. PFS project is planning to apply for SSP of ~300 nights.}


\newdualentry{STARS}
{STARS}
{Subaru Telescope ARchive System}
{Data Archive system at Subaru Telescope (in Hilo). Images sent to gen2 are archived to STARS.}


\newdualentry{sth}
{sth}
{SNMP trap handler}
{ICS actor to handle alerts (trap) sent via SNMP from PFS control infrastructures.}


\newdualentry{STS}
{STS}
{Subaru Telemetry System}
{Internal service for telemetry of Subaru instruments.}


\newdualentry{TCS}
{TCS}
{Telescope Control System}
{Computing system to control Subaru Telescope. TCS has hierarchy structure.}


\newdualentry{TEM}
{TEM}
{Telemetry Monitor}
{PFI telemetry system, Temperature, humidity, flow rate and sound are monitored. ASIAA has a responsibility for its development.}


\newdualentry{TSC}
{TSC}
{Telescope Supervise Computer}
{Computers which supervise the whole computers controlling telescope such as movement, mirror support, environment.}


\newdualentry{TUE}
{TUE}
{Top-End Unit Exchanger}
{Mechanism to exchange the optics (prime focus instrument and secondary mirrors) at top ring of Subaru telescope.}


\newdualentry{TUE-IR}
{TUE-IR}
{Top-End Unit Exchanger IR-side}
{The floor name of the dome building which used to be used for IR4.}


\newdualentry{UFC}
{UFC}
{USB to Fiber Convertor}
{Device used for AG cameras. It converts USB signal from AG cameras to fiber signal from cable wrapper (hand hence from Subaru). It is accommodated in Auxiliary Power/control Electronics (APC).}


\newdualentry{ULS}
{ULS}
{Upper/Lower Link structure}
{Structure to connect the cable wrapper assembly (CWA) and the instrument rotator. ULS also accommodates three electronic boxes (PRE-A, PRE-B, and APC).}


\newdualentry{WFC}
{WFC}
{Wide Field Corrector}
{Corrector lens for Subaru prime focus. Total F ratio if F/2.8, and FoV is approximately 1.5 degree in diameter. PFS shares WFC with HSC.}

